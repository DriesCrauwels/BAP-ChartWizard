\pagenumbering{roman}

\selectlanguage{dutch}
\frontmatter
\thispagestyle{empty}   


\chapter{Dankwoord}
\vspace{0.35in}

Graag zou ik eerst en vooral mijn medestudent Sam De Jongh willen bedanken. Samen hebben wij 4 maanden gewerkt bij Inventive Designers. Dankzij hem heb ik nooit de moed verloren en was het aangenaam werken. Onze interne promoter Erik Vanherck heeft ons beide goed begeleid en interessante opdrachten gegeven. Ten slotte bedank ik ook nog graag Tom Peeters en Tim Dams voor hun feedback en snelle response op vragen en opmerkingen over onze bachelorproef.


\begin{flushright}{\emph{Antwerpen, \today \\
Dries Crauwels}}
\end{flushright}


\chapter{Abstract}
Big data is vandaag in de mode. De vraag ernaar stijgt en meer en meer bedrijven stellen hun data publiekelijk ter beschikking. Big data draait om grote hoeveelheden, de snelheid waarmee de data binnenkomt en opgevraagd kan worden, en de diversiteit van de data. Denk maar aan landkaarten, data afkomstig van CERN, internetverkeer enzovoorts. De technologie hiervoor bestaat al enige tijd maar de mindset is nu helemaal anders. Alle data op een hoop gooien en iedereen kan er gebruik van maken, dat is nieuw. Echter niemand heeft iets aan een berg bits en bytes. We moeten iets met deze data doen.

We gaan populaire datasets op het web(JSON,CSV,XML) visualiseren in een web applicatie. De gebruiker zal zijn eigen data kunnen uploaden en de website zal deze voor hem visualiseren. Daarna zal je de visualisatie kunnen exporteren om op te nemen in jouw publicatie. Dit kan zeer nuttig zijn voor onder andere tijdschriften en kranten die graag info beter naar de lezer willen overbrengen. 

Visualiseren van data kan heel ver gaan. Van infographics over de ganse wereld tot de volledige opbouw van een genoom. Deze applicatie zal dan ook nooit volledig klaar zijn en men zal er blijven aan moeten werken om nieuwe soorten data op verschillende manieren te visualiseren. Deze website is een eerste stap in de goede richting. Ik streef er dan ook naar om onderhoudbare en universele code te schrijven zodat andere mensen er makkelijk verder mee kunnen en functionaliteit kunnen toevoegen.

De volledige code zal bestaan uit jQuery, Javascript, HTML 5 en CSS 3 en de D3 javascript library. Deze D3 library is de hart van de applicatie. Ze is geschreven specifiek om makkelijker SVG elementen in een HTML pagina te genereren. Deze SVG elementen vormen dan een grafische voorstelling van de data.
De code zal opgebouwd zijn uit \'{e}\'{e}n library die alle code bevat voor het tekenen van de data in HTML.De rest van de code zal de look and feel van de applicatie zijn en de gebruikers interface. 

Totzover heeft de applicatie 4 werkende wizards voor 4 verschillende charts en \'{e}\'{e}n grotere wizard voor een infographic waarin de 4 kleinere wizards in zijn verwerkt.


% Inhoudstabel invoegen
\tableofcontents

% Lijst met alle tabellen invoegen
%\listoftables

% Lijst met alle figuren invoegen
\listoffigures