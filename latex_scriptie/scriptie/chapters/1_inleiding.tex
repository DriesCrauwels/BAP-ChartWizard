\chapter[Inleiding]{Inleiding}
\label{chap_inleiding}

De opdracht bestaat er in om een interfaces te coderen en te designen op een website waarnaar de gebruiker zijn eigen data kan uploaden om  er een visuele voorstelling van te krijgen. 

Hiervoor moet je al goed weten in welke soort formaten de meeste data voorkomen en moet de applicatie de gebruiker zelf opleggen om een bepaalde syntax te gebruiken in zijn data. Men kan de code aanpassen om meerdere syntaxen uit te lezen maar dit vraagt meer werk dan dat de gebruiker wat eigenschappen aanpast in zijn bestand.

Op het web wordt vooral JSON CSV en XML data gebruikt als datasets. De gebruiker moet \'{e}\'{e}n van deze bestandssoorten kunnen uploaden naar de server en de applicatie leest deze dan uit en tekent SVG elementen. Deze SVG elementen vormen dan een chart die men kan exporteren als een SVG, PNG bestand of HTML code.


\section{Programmeertalen}
Aangezien we met de D3 library een web applicatie moeten ontwikkelen en dit een javascript library is, is het vanzelfsprekend voor een programmeur om de volgende programmeertalen te gebruiken als ondersteuning voor deze opdracht.
\subsection{HTML 5}
\begin{quote}
"HTML or HyperText Markup Language is the standard markup language used to create web pages.

HTML is written in the form of HTML elements consisting of tags enclosed in angle brackets (like \textless html\textgreater ). HTML tags most commonly come in pairs like \textless h1\textgreater and \textless /h1\textgreater , although some tags represent empty elements and so are unpaired, for example \textless img\textgreater . The first tag in a pair is the start tag, and the second tag is the end tag (they are also called opening tags and closing tags).

The purpose of a web browser is to read HTML documents and compose them into visible or audible web pages. The browser does not display the HTML tags, but uses the tags to interpret the content of the page. HTML describes the structure of a website semantically along with cues for presentation, making it a markup language rather than a programming language."
\end{quote}
\textit{Source: HTML - \url{https://en.wikipedia.org}}

\subsection{CSS 3}
\begin{quote}
"Cascading Style Sheets (CSS) is a style sheet language used for describing the look and formatting of a document written in a markup language. While most often used to style web pages and interfaces written in HTML and XHTML, the language can be applied to any kind of XML document, including plain XML, SVG and XUL. CSS is a cornerstone specification of the web and almost all web pages use CSS style sheets to describe their presentation.

CSS is designed primarily to enable the separation of document content from document presentation, including elements such as the layout, colors, and fonts. This separation can improve content accessibility, provide more flexibility and control in the specification of presentation characteristics, enable multiple pages to share formatting, and reduce complexity and repetition in the structural content (such as by allowing for tableless web design)."
\end{quote}
\textit{Source: Cascading Style Sheets - \url{https://en.wikipedia.org}}

\subsection{JavaScript}
\begin{quote}
"JavaScript is a prototype-based scripting language with dynamic typing and has first-class functions. Its syntax was influenced by C. JavaScript copies many names and naming conventions from Java, but the two languages are otherwise unrelated and have very different semantics. The key design principles within JavaScript are taken from the Self and Scheme programming languages. It is a multi-paradigm language, supporting object oriented, imperative, and functional programming styles."
\end{quote}
\textit{Source: JavaScript - \url{https://en.wikipedia.org}}

\subsection{jQuery}
\begin{quote}
"jQuery is a fast, small, and feature-rich JavaScript library. It makes things like HTML document traversal and manipulation, event handling, animation, and Ajax much simpler with an easy-to-use API that works across a multitude of browsers. With a combination of versatility and extensibility, jQuery has changed the way that millions of people write JavaScript."
\end{quote}
\textit{Source: What is jQuery? - \url{http://jquery.com}}

\subsection{D3}
\begin{quote}
"D3.js is a JavaScript library for manipulating documents based on data. D3 helps you bring data to life using HTML, SVG and CSS. D3's emphasis on web standards gives you the full capabilities of modern browsers without tying yourself to a proprietary framework, combining powerful visualization components and a data-driven approach to DOM manipulation."
\end{quote}
\textit{Source: D3 - \url{http://d3js.org}}


